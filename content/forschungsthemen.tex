%!TEX root] ../report.tex

\section{Forschungsthemen}

\subsection{Relative Age Effect}

Abweichung der Geburstage einer Stichprobe von der eigentlich zu erwartenden Verteilung. Meistens eine Häufug der Geburstage zum Beginn der Selektionsperiode.

Entstehung: Junge Athlethen konkurieren innerhalb ihrer Altersklassen. Frühgeborene (in einem Selektionszeitraum) haben mit höherer Wahrscheinlichkeit einen größeren Erfahrungshorizont, anthropologische und physische Vorteile. Daher auch potentiell besseres Abschneiden bei initialen Diagnostiken und dadurch höhere Wahrscheinlichkeit ``entdeckt'' zu werden.

Gleichzeitig gibt es einen selbstverstärkenden Effekt. Mehr positive Erlebnisse für Frühgeborene führen zu höherer Motivation, mehr Selbstvertrauen und dadurch auch höhere Chancen auf Fördermaßnahmen.
Insgesamt führt dies zu Talenten mit zu geringen Chancen auf Förderung und der Verschwendung von Ressourcen für eher untalentierte Athleten.

Weitere Fakten:
\begin{itemize}
    \item Tritt vorallem in körperbetonten sportarten auf
    \item betrifft Männer stärker als Frauen
    \item Phänomen seit den 60er-Jahren bekannt
    \item In fortgeschrittenen Selektionsebenen geringer
\end{itemize}

In einer Studie von Nachwuchsfußballern konnte der RAE sehr klar nachgewisen werden (Median Geburtsdatum: U15 1.3., U17 21.3., U19 10.4., ...). Der RAE wächst sich aber klar raus, mit jedem Jahr älteren relativen Alters sind die Spieler 6,2 Tage später geboren.

Diskussion:
\begin{itemize}
    \item RAE umso größer, je unterschiedlicher der Stand der körperlichen Entwicklung der Sportler (Kein RAE in BL und A-Nationalmannschaften)
    \item RAE umso größer, je größer die Konkurrenz
    \item Gründe für die Selektierung relativ älterer Spieler: Momentane Leistungsfähigkeit entscheidend für Auswahl, Erfolg des Trainers für diesen wichtiger als das Potential des Spielers
    \item Gründe für die Abnahme des RAE mit dem Alter: Karrieren der früh im Jahr geborenen Spieler sind kürzer, da sie durch exzessives Training von klein auf “verheizt” wurden, Entwicklung besserer fußballerischer Fähigkeiten der Spätgeborenen aufgrund ihrer körperlichen Defizite, die sie im Erwachsenenalter aufgeholt haben
\end{itemize}

Lösung: Berücksichtigung des fußballerischen Potentials statt der momentanen
Leistung bei Förderung talentierter Nachwuchsspieler. Erlangen von Aufmerksamkeit über die RAE-Problematik bei Trainern, Ändern der Trainer- Philosophie: „Säen statt Ernten“. Einführen von alternierende Stichtagen

\subsection{Schiedsrichterforschung}

Ursachen von Fehlentscheidungen: ``Launen der Wahrnehmung und des Gewissens''
\begin{description}
    \item [Wahrnehmung] Triviale räumliche Gründe, Physisch bedingte Wahrnehmungsverzerrungen
    \item [Gewissen] Kriminelle Motive, Bias, unbewusste Vorurteile
\end{description}

Wahrnehmungsverzerrung:
\begin{description}
    \item[Optical Error Hypotheses] Modell unter Einbezug der Positionierung des Linienrichters, Kann als Erklärung einer Vielzahl von Abseits-Fehlentscheidungen herangezogen werden
    \item [Flas-Lag Effect] Bekanntes Phänomen aus Labortests: Für sich schnell fortbewegende Objekte verschiebt sich die Wahrnehmung in Fortbewegungsrichtung.
\end{description}

Bias:
\begin{itemize}
     \item Homebias: Nachgewiesen für eine Reihe von Sportarten und Entscheidungen, Hoher Anteil am Phänomen ``Heimvorteil''
     \item Bevorteilung von Spielern gleicher ethnischer Gruppe (Nachgewiesen in den USA u.a. in Baseball und College Basketball)
     \item Einfluss von Standings und Reputation: Richtung I: Bevorzugung renommierter Athleten/ Teams, Richtung II: Wahrnehmungsverzerrung durch zugeschriebene Charaktereigenschaften
 \end{itemize}

Technological Umpiring Aids:
\begin{itemize}
    \item Technologische Unterstützung von Schiedsrichtern
    \item Eingesetzt zur Reduzierung von Fehlentscheidungen
    \item Einführung sollte begleitet werden von umfassenden Evaluationsprozess: Technisch (Genauigkeit, Sicherheitsaspekte), Meinungsbild aller Stakeholder, Einfluss auf das Spiel
\end{itemize}

Torlinientechnologie:
\begin{itemize}
    \item Häufigkeit: Geringer Anteil torkritischer Entscheidungen sind torlinien-kritisch (5\%), Torlinienkritische Entscheidungen welche nur mit TLT aufgeklärt werden könnten sind sehr selten (1 pro Spieltag)
    \item Argumente für Videobeweis: 77\% der torlinien-kritischen Ereignisse könnten aufgeklärt werden, 86\% aller kritischen Tore (Abseits, Hand, Foul) könnte aufgeklärt werden. Usus in anderen Sportarten
    \item Argumente gegen Videobeweis: Kann nicht alle kritischen Torsituationen aufklären, Erfordert Regeländerung/ -anpassungen
\end{itemize}

Freistoßspray:
\begin{itemize}
    \item Fragestellung: Führt die Einführung des Freistoßsprays zu weniger Regelverstößen und dementsprechend weniger Verwarnunge und Freistoß-Wiederholungen?
    \item Spray wird vorallem Zentral und in Tornähe eingesetzt
    \item Kein Rückgang der Anzahl an Freistößen mit Regelverstößen, aber zumindest Rückgang des Ausmaß an Regelverstößen
    \item Keinen positiven Einfluss auf den Erfolg von Freistößen
    \item Keine Auswirkung auf die Abschätzung des Abstandes durch den Schiedsrichter
    \item Regelverstöße werden weiterhin kaum sanktioniert
\end{itemize}
