%!TEX root = ../report.tex

\section{Technik}

Generell geht es darum Bewegungsabläufe in Bezug auf Geschwindigkeit und/oder Genauigkeit zu optimieren. Dadurch lassen sich Konditionelle Fähigkeiten optimal ausnutzen, Belastungen reduzieren und das Verletzungsrisiko senken.

\subsection{Begriffe \& Systematik}

Technik ist eine Sammelbezeichnung für eine Reihe technischer Fertigkeiten eines Sportlers/einer Sportart. Eine technische Fertigkeit ist eine erprobte, zweckmäßige und effektive Bewegungsfolge zur Lösung einer definierten Aufgabe in Sportsituationen. Beispiel: Pritschen ist erprobte Bewegungsabfolge der Aufgabe ``Zuspielen'' und teil der technischen Fertigkeiten eines Volleyballers.

Eigenschaften der Technik:
\begin{itemize}
    \item Individualitätseigenschaft: Eindeutige Identifizierung von Weltklasseathleten auf Basis ihrer individuellen Technik
    \item Stabilitätseigenschaft: Technische Fertigkeiten sind hochgeübte Bewegungsfolgen die erst nach Jahren beherrscht werden
    \item Variabilitätseigenschaft: Biomechanische Messungen zeigen immer Variabilitäten, Keine Bewegung entspricht exakt einer anderen und Fähigkeit zur Anpassung der Bewegung während der Ausführung wichtig
\end{itemize}

Es gibt verschiedene Möglichkeiten die Technik zu unterteilen.
Beispiele:
\begin{itemize}
    \item Elementare Fähigkeiten vs. komplexe Sportspezifische Fertigkeiten
    \item Umwelt variabel / konstant vs. mit / ohne Zeitruck
    \item  Sportartspezifische Systematiken (Volleyball: Abwehr, Aufbau, Angriff)
\end{itemize}

Systematiken:
\begin{itemize}
    \item Sporttechnisches Leitbild:
    \begin{itemize}
        \item Idealtechnik: Optimale Bewegungsfolge zur Lösung der Bewegungsaufgabe
        \item Zieltechnik: Für ein Individuum optimale und anzustrebende Bewegungsfolge
    \end{itemize}
    \item Bewegungsnormen:
    \begin{itemize}
        \item Idealnorm: Wissenschaftlich optimale Bewegungsfolge oder Lösungen der Weltbesten
        \item Funktionale Norm: Notwendige Anforderung, um ein Ziel zu erfüllen
        \item Statistische Norm: Wie macht‘s eine vergleichbare Stichprobe?
    \end{itemize}
    \item Technikerwerbstraining: Neulernen bis Automatisierung des dynamischen Optimums
    \item Technikvariationstraining: Varianten und ihr situationsgerechter Einsatz
    \item Technikanpassungstraining: Anpassung an variable Umwelt (Gelände, Raum, Zeit)
    \item Technikabschirmungstraining: Abschirmen gegen  Ermüdung, Gegner und psychischen Druck
\end{itemize}

\subsection{Determinanten}



\subsection{Trainingsmethoden}

\subsection{Trainingsinhalte}

\subsection{Anwendung}

\subsection{Diagnostik}


