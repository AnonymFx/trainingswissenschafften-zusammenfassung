%!TEX root] ../report.tex

\section{Trainingsmodelle}

Generell:
\begin{description}
    \item[kurzfristig] Leistungsveränderungen durch optimale Gestaltung der Einzelreize anstoßen
    \item [mittelfristig] Leistungsfähigkeit steigern und herausbilden der optimalen Form für einen bestimmten Zeitraum
    \item [langfristig] Leistungspotential langfristig ausschöpfen
\end{description}

\subsection{kurzfristig}

Grundlegende Begriffe:
\begin{description}
    \item [Homöostase] dauerhaft aufrechthaltbarer Gleichgewichtszustand in Ruhe
    \item [Heterostase] Störung der Homöostase
    \item [Belastung] objektiv von Außen auf den Organismus wirkende Faktoren Umstellung] Zustandsveränderung der Funktionssysteme im Rahmen ihres Regulationsbereiches durch Belastung (auch Aktivierung)
    \item [Beanspruchung] individuelle Reaktion des Organismus auf eine Belastung
    \item [(Max) Steady State] zeitlich begrenzt aufrechthaltbare (maximale) Leistung während der Belastung
    \item [Ermüdung] reversible Leistungsminderung durch Beanspruchung
    \item [Regeneration] Prozesse zur Wiederherstellung der Homöostase
    \item [Superkompensation] überschießende Wiederherstellung der Leistungsfähigkeit
    \item [Adaptation] zeitlich stabile, reversible Steigerung der Leistungsfähigkeit durch Beanspruchung
    \item [Deadaptation] Rückbildung der Leistungsfähigkeit durch fehlende Beanspruchung
\end{description}
