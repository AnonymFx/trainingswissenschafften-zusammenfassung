%!TEX root] ../report.tex

\section{Wettkampfmodelle}

\subsection{Definition und Wesen}

``Ein sportlicher Wettkampf ist ein Leistungsvergleich nach den festgelegten Regeln einer Sportart zwischen einzelnen Sportlern oder Mannschaften zum Zwecke der Ermittlung eines Siegers (einer Rangfolge).''

Natur des Wettkampfes:
\begin{description}
    \item[Ausdauer- und Kraftsportarten] taktisch geschickter Einsatz der führenden Leistungskomponente
    \item [Technisch-kompositorische Sportarten] Fertigkeiten in höchster Qualität präsentieren
    \item [Sportspiele und Kampfsportarten] Auseinandersetzung mit dem Gegner gewinnen
\end{description}

Gründe für sportlichen Wettkampf:
\begin{itemize}
    \item Leistungssport: Individuelle Höchstleistung, Absoluter sportlicher Erfolg
    \item Breitensport / Wettkampfsport: Sportliche Erfolge innerhalb vonLeistungsklassen der Sportart, Teils ohne Wettkämpfe
    \item Schulsport: Wichtige pädagogische Kategorie, Wichtiger Aspekt des Sporttreibens
    \item Nachwuchssport / Talentförderung: Langfristiger und altersgerechterAufbau sportlicher Leistung
\end{itemize}

Funktionen sportlicher Wettkämpfe:
\begin{itemize}
    \item Kulturell-erzieherische Funktion:  Fair-Play, Aneignung vonCharaktereigenschaften
    \item Soziale Funktion:  Ausbildung von Zusammengehörigkeitsgefühl
    \item Kommerzielle Funktion:  Profisport, materieller Gewinn
    \item Unterhaltungsfunktion:  Medienunternehmen
    \item Politische Funktion:  Z.B. Symbolcharakter für dieLeistungsfähigkeit einer Nation
\end{itemize}

\subsection{Wettkampfsysteme}

Gibbet alles mögliche, Rundensysteme, Ligen-Systeme und Knockout...

Setzung ist wichtig, um das frühzeitige Ausscheiden eigentlich leistungsstärkerer Teilnehmer zu vermeiden. Dabei kann man maximalen (1 vs.\ 8) oder minimalen (1 vs.\ 5) Vorteil für den Favoriten geben.

\subsection{Trainingswissenschaft und Wettkampf}

Der Wettkampf wurde erst recht spät (Mitte 90er Jahre) wissenschaftlich betrachtet. Es geht um die wissenschaftliche Fundierung von Maßnahmen, die sich unmittelbar auf einen bestimmten Wettkampf beziehen (``Wettkampfsteuerung'', ``Coaching''). Grundsätzlich stellt der Wettkampf Anforderungen an die Leistungsfähigkeit und ist gleichzeitig ein indikator für sie. Er stellt außerdem die Motivation und Zielgröße für das Training und kann als Trainingsmitel verwendet werden. Unter Umständen ist er auch Indikator für den Trainingserfolg.

\begin{tabular}{m{0.5\textwidth} | m{0.5\textwidth}}
    Training & Wettkampf \\ \hline
    Leistungsfähigkeit wird langfristig aufgebaut & Leistung wird kurzfristig realisiert \\
    großzügige Zeitlimits für Bewegungen und Handlungen & Begrenzte Zeitlimits durch die Wettkampfregeln \\ Starke Einflussnahme durch Trainer &  Beschräkte Trainerbeeinflussung, verstärkte  Selbststeuerung \\
    In der Regel keine außergewöhnliche mentale Belastung & erhöhte nervliche Spannung, hohe Emotionalität \\
    vergleichsweise gut plan- und steuerbar & wenig vorhersagbar, spontane Wendungen, Einzelfallereignisse \\
\end{tabular}

Aus trainingsmethodischer Sicht ist der Wettkampf die Zielgröße des sportlichen Trainings als Möglichkeit die erworbenen Fähigkeiten zu zeigen.
ER wird außerdem als methodische Mittel zur Entwicklung der wettkampfspezifischen Leistungsfähigkeit benutzt, weil Wettkampfbedinungen im Training nur bedingt simuliert werden können.

\subsection{Wettkampfsteuerung}

Alle Trainingsmaßnahmen, die sich unmittelbar auf einen bestimmten
Wettkampf beziehen (=Coaching).
Unterscheidung nach Zeitpunkt (Vor, Während, Nach dem Wettkampf) und Unterstützungsebene (Emotionale, Soziale, Organisatorische oder Leistungsebene.)

Vor dem Wettkampf analysierst du halt vergangene Wettkämpfe, entwickelst dir eine coole Strategie und implementierst die dann für den Wettkampf.
Dabei spielen die Eigen- und Gegneranalyse und die Trainerphilosophie eine Rolle für die Strategiewahl.

Die Wettkampflenkung sind die Maßnahmen des Trainers während des Wettkampfes. Dabei sind die Phasen identisch zur Vorbereitung (Analyse, Strategieentwicklung, Implementation), herausfordernd sind aber die Rahmenbedingungen (Datenerfassung und -verarbeitung, beschränkte Interventionsmöglichkeiten). Zunehmend wird auf technologische Unterstützung zurückgegriffen.

Nach dem Wettkampf geht es zunächst um emotionale (``runterkommen'') und physiologische (Einleitung der Regenerationsphase) Unterstützung. Mit Zeitlichem Abstand dann eher um informatorische Unterstützung durch Bewertung des Wettkampfverlaufs und Ableitung von Konsequenzen für die Trainingsplanung. Das Ziel ist eine Rückmeldung für Athleten zu geben und das Training  und die Strategieentwicklung zu verbessern.

\subsection{Wettkampfdiagnostik}

Die Wettkampfdiagnostik sollte aufgrund der Nähe zum sportlichen Erfolg eigentlich der wichtigste Bereich für die trainingswissenschaftliche Leistungsdiagnostik sein.
Erfolg ist zentrales Kriterium im Hochleistungssport und Erfolge werden im
Wettkampf erzielt (oder nicht!). Das Wettkampfverhalten ist näher am Erfolg als z.B. Leistungsvoraussetzungen. Wenn Wettkampfdiagnostik das Wettkampfverhalten nur geringfügig verbessern hilft, dann verbessert dies unmittelbar unsere Erfolgsaussichten!

In der Realität bezieht sich die praktische Leistungsdiagnose oft nur auf die Diagnostik von Leistungsvorrausetzungen. In manchen deutschen Verbänden wird Wettkampfdiagnostik nicht als Aufgabe wahrgenommen und oft wird unter ``Spielanalyse'' nur das sequentielle Betrachten von Wettkampfvideos mit ad hoc-Kommentaren des Trainers verstanden.

Es gibt diverse verschiedene Lösungen zur Positionserfassung, dabei kann auf Radar (teuer aber genau, aktive sender), GPS (low-budget möglich, antennen nötig, nicht indoor geeignet) oder Bilderkennung (low-budget möglich, keine physisch markierungen nötig, potentiell 3d möglich).
Die gewonnenen Daten können auf verschiedene Arten ausgewertet werden: Heatmaps, taktische konfigurationen, ...


