%!TEX root = ../report.tex

\section{Schnelligkeit}
Schnelligkeit ist die Fähigkeit in ermüdungsfreiem Zustand mit möglichst kurzen zeitlichem Abstand auf einen Reiz zu reagieren oder zu agieren.
\begin{figure}[H]
  \centering
  \includegraphics[width=.5\textwidth]{pictures/schnelligkeit_overview.png}
  \caption{Überblick Schnelligkeit}
\end{figure}
\begin{figure}[H]
  \centering
  \includegraphics[width=.5\linewidth]{pictures/schnelligkeit_determinanten.png}
  \caption{Determinanten der Schnelligkeit}
\end{figure}

\subsection{Begriffe, Systematik und Determinaten}
\subsubsection{Reaktionsschnelligkeit}
\begin{description}
  \item[Reaktionsfähigkeit] ist die psychophysische Fähigkeit auf Reize schnell zu reagieren.
  \item[Elementare Reaktionsschnelligkeit] Kleinmotorische Bewegungsantworten auf einfache Reize
  \item[Komplexe Reaktionsschnelligkeit] Großmotorische Bewegungsantworten \& komplexe (Wahl-) Reaktionen.
\end{description}

\textbf{Modell der Reaktionsgeschwindigkeit}\\
\begin{figure}[H]
  \centering
  \includegraphics[width=.5\textwidth]{pictures/reaktionsgeschwindigkeit_modell.png}
  \caption{Modell der Reaktionsgeschwindigkeit}
\end{figure}
\begin{description}
    \item[Übertragung bis Rezeptor] hängt vorwiegend von den Eigenschaften des Rezeptors ab (1-20ms).
    \item[Restliche ÜBertragung] wird von den Eigenschaften des Nervensystems beeinflusst (Rezeptor $\rightarrow$ ZNS: 1-100ms, ZNS $\rightarrow$ Effektor: 10-20ms).
    \item[Verarbeitung im ZNS] wird beeinflusst von psychischen Eigenschaften wie Aufmerksamkeit und Wahrnehmung und kann von 70 bis 300ms dauern.
    \item[Verarbeitung in den Effektoren] Abhängig von neuromuskulären Eigenschaften wie Fastetypzusammensetzung oder inter-/intramuskuläre Koordination
\end{description}
Die gesammte Übertragungszeit beträgt in etwa 112 - 510ms.

\subsubsection{Bewegungsschnelligkeit}
Bewegungsschnelligkeit ist die Fähigkeit, Bewegungen in höchster Geschwindigkeit oder kürzester Zeit auszuführen.
\begin{figure}[H]
    \centering
    \includegraphics[width=.7\textwidth]{pictures/bewegungsgeschwindigkeit_struktur.png}
    \caption{Struktur der Bewegungsschnelligkeit}
\end{figure}

Die Bewegungsschnelligkeit hängt von drei Bereichen ab:
\begin{description}
    \item[Neuromuskuläres System] Neuronale Steuer- \& Regelprozesse, Reaktionsgeschwindigkeit, inter/intra-muskuläre Koordination,\ldots
    \item[Psychisches System] Konzentration, Wahrnehmung, Motivation, \ldots
    \item[Tendomuskuläres System] Querschnittsfläche FT-Fasern, Stiffness, Viskosität, \ldots
\end{description}

\subsection{Trainingsmethoden}
Schnelligkeit ist schlechter zu trainieren als Ausdauer oder Kraft.\\
Zu trainierende Determinanten sind:
\begin{figure}[H]
    \centering
    \includegraphics[width=.7\textwidth]{pictures/schnelligkeit_trainierbare_determinanten.png}
    \caption{Trainierbare Determinanten}
\end{figure}

\paragraph{Belastungsnormative}
\begin{description}
    \item[Intensität] Bewegung auf maximaler Geschwindigkeit
    \item[Dauer] 8-10s, Abbruch bei Erschöpfung
    \item[Pause] bis vollständig regeneriert, Schnelligkeitstraining vor anderen
    \item[Durchführung] Mit maximaler Konzentration \& Willen, mit Aufwärmen
\end{description}

\subsubsection{Reaktionsbereitschaft}
Training beinhaltet einfache Reaktionen (Pfiff, (Nacken- :P) Klatschen) auf akustische, visuelle oder taktile Reize.

\subsubsection{Entscheidungsvorgänge}
Eine Wahlreaktion wird vor einer motorischen Reaktion eingefordert.
Auch die richtige Wahrnehmung ist wichtig, Signal wird erschwert zu verstehen (leiser Pfiff = noop, lauter Pfiff = go)

\subsubsection{Neuromuskuläre Ansteuerung}
Grundbewegungen schnell realisieren. Es wird unterschieden zwischen zyklischen (wiederholenden) und azyklischen (einmaligen) Bewegungen.
\paragraph{Geschwindigkeitsbarriere} Häufige Realisation der Maximalgeschwindigkeit führt zu einer Verfestigung der neuromuskulären Ansteuerung (Reizleitungswege, Innervationsmuster).
Gegenmaßnahmen sind nur wenig Wettkampfsimulationen mit Maximalgeschwindigkeit (1/Woche), Variation der Bewegung und Entwicklung der elementaren Fähigkeiten
\paragraph{Erleichterte Bedingungen} Erleichtern der Übungen durch verringern des Wiederstands oder Körpergewichts um Geschwindigkeit zu erhöhen.
Beispiele sind Treppenläufe bergab (zyklisch) oder leichtere Wurfgeräte (azyklisch).
\paragraph{Räumliche Zwänge} Erzwingen einer höheren Frequenz durch Einengungen (z.B. Fesseln).
\paragraph{Mentales Training} Der Carpenter-Effekt (Denken an eine Bewegung bewirkt diese in einem abgeschwächtem Maß) wird dazu genutzt mit Metaphern in der Übungsbeschreibung eine Aktion hervorzurufen (``Springe wie ein Frosch'').
\paragraph{Elektromyostimulation} Elektrische Stimulation auf das Innervationsmuster. Die nicht belegte Hypothese ist dass dadurch eine bessere Rekrutierung motorischer Einheiten erfolgt und das bestehende Innervationsmuster ``überschrieben'' wird.

\subsubsection{Technik}
Technikübungen sind disziplinspezifisch, schnell ausgeführte Bewegungen die meist überbetont werden.
Kombinationen aus einzelnen Aktionen und Übungen sind möglich.

\subsubsection{Schnellkraft}
Eine Erschwerung der Bedingungen (z.B. Laufen mit Zugschlitten) führt zu einem Training für Kraft und Schnelligkeit.
Dadurch ist allerdings die maximale Geschwindigkeit nicht mehr zu erreichen.

\subsection{Anwendung}
Schnelligkeit wird eher selten explizit Trainiert.
Stattdessen existieren Schnittstellen zu anderen Trainingsarten wie Koordinationstraining (elementare Bewegungsschnelligkeit) und Techniktraining (komplexe Bewegungsschnelligkeit).\\
Früh gesetzte Schnelligkeitsreize im Kindesalter haben hohe Auswirkungen auf die maximal erreichbare Schnelligkeit im Erwachsenenalter.
\begin{enumerate}
    \item Elementare neuromuskuläre Ansteuerung im Alter von 7-9 Jahren
    \item Komplexe intermuskuläre Koordination von 10-12 Jahren
    \item Kraft- und Ausdauerkomponenten mit 12-14 Jahren
\end{enumerate}
