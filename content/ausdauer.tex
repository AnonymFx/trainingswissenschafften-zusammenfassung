%!TEX root = ../report.tex

\section{Ausdauer}
Definition: Ermüdungswiderstandsfähigkeit (konditionell und informationell) und Regenerationsfähigkeit

\subsection{Systematik}
\subsubsection{Belastungsdauer}
\begin{itemize}
  \item Kurzzeitausdauer (<35s)
  \item Mittelzeit-Ausdauer(35s - 8min)
  \item Langzeitausdauer (> 8min): LZA I: 8-30min, LZA II: 30min - 3h, LZA III: 3-9h, LZA IV: >9h
\end{itemize}

\subsubsection{Allgemeine vs spezielle Ausdauer}
\begin{itemize}
  \item Allgemeine Ausdauer: Grundlage für Regeneration, Training und Erholung und Voraussetzung für spezielles Training
  \item Spezielle Ausdauer: Wettkampfspezifische Ausdauer-Anforderungen (wichtiger für sportlichen Erfolg)
\end{itemize}

\paragraph{KsA: Koeffizient der speziellen Ausdauer}
= durchschnittliche 100m-Zeit obere Nachbarstrecke / durchschnittliche 100m-Zeit untere Nachbarstrecke

\subsection{Energiebereitstellung}
\paragraph{Mechanismen}
\begin{itemize}
  \item Systematik: Je nach Dauer und Intensität der Belastung wird Energie aus verschiedenen Substanzen gewonnen.
  \item Energiegewinnung aus Substrat:
    \begin{itemize}
      \item Energiefluss: Rate, mit der Energie zur Verfügung gestellt werden kann.
      \item Kapazität: Energievorrat
    \end{itemize}
  \item Problem: Maximale Energie nur kurzfristig verfügbar\\
    Dauerleistungsgrenze: ca 40%
  \item Energiearten
    \begin{itemize}
      \item alaktazid
      \item laktazid
      \item aerob
    \end{itemize}
\end{itemize}

\subsection{Determinaten der Ausdauer}
\begin{tabular}{l | l | l}
                          & Intensiv                                                         & Extensiv \\ \hline
     Energiespeicher      & Phosphat                                                         & Glykogen \\ \hline                          
     Enzymaktivityät      & Phosphatstoffwechsel Laktatabbau und -toleranz                   & Kohlenhydrat- und Fettstoffwechsel \\ \hline
     Muskulatur           & vortriebrelevante Muskulatur                                     & Haltearbeit verrichtete Muskulatur \\ \hline
     Sauerstoffversorgung & Schlagvolumen, Kapillarisierung der Arbeitsmuskulatur, Blutmenge \\ \hline
     Qualität der Technik & Bewegungsökonomie \\ \hline                                               
     Psychische Eigensch. & Durchhaltevermögen, "Stehvermögen", "mentale Härte" \\
\end{tabular}<++>
