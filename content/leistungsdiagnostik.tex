%!TEX root = ../report.tex
\section{Leistungsdiagnostik}
Leistungsdiagnostik ist die Feststellung der Wettkampfleistung, von Wettkampf-Teilleistungen und von Leistungsvoraussetzungen.
Die Messung der Leistungsvorraussetzungen erfolgt über motorische Tests, leistungsphysiologische Verfahren (Laktat), biomechanische Verfahren (Laufanalysen) und psychologische Verfahren.
Die Messung der Wettkampfleistung über biomechanische Messungen (von einzelnen Läufen), Wettkampfdokumentationen und Spielbeobachtungen.

\subsection{Theoretische vs.\ Praktische Leistungsdiagnostik}
\paragraph{Theoretische Leistungsdiagnostik (TLD)}
Leistungsdiagnostik zum Zweck der Erstellung allgemeiner Aussagen über die Leistungsstruktur einer Sportart.\\
Typische TLD Fragestellungen:
\begin{itemize}
  \item Allgemeine Modellierung der sportlichen Leistung
  \item Sportartspezifische Modellbildung
  \item Alters- und Geschlechtsabhängigkeit der Leistungsstruktur
  \item Abhängigkeit der Leistungsstruktur vom Leistungsniveau
  \item Erstellung statistischer Normen
\end{itemize}
\paragraph{Praktische Leistungsdiagnostik (PLD)}
Leistungsdiagnostik im Kontext eines Trainingsprozesses.
PLD diagnostiziert die Wettkampfleistung und den Leistungszustand von Mannschaften und Athleten.
PLD ergängzt die Methoden von TLD mit qualitativen Analyzen von Einzelfällen.
Die Unterschiede von TLD und PLD liegen vorallem darin, dass sich PLD eher mit Einzelfällen und TLD eher für allgemeine, statistische Aussagen interessiert.
Die beiden Methoden beeinflussen sich allerdings: TLD generiert Hintergrundwissen für PLD und PLD liefert die VOrgaben für TLD (Hypothesen, Inspiration für Modelle).
