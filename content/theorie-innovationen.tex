%!TEX root] ../report.tex

\section{Theorie-Innovationen}

\subsection{Expertisenforschung}

Expertise: Dauerhafte Leistungsexzellenz in einer Domäne (Gesellschaftlich definiertes Aktionsfeld). Die Leistungsexzellenz wird dabei immer relativ zu einem Bezugssystem gemessen, traditionell zum Weltniveau.

Ursprung bei K. Anders Ericsson, Psychologe aus Florida, untersuchte die Laufbahn von Experten aus den Domänen Schach, Mathematik und Musik. Aussagen in der ersten harten Fassung:
\begin{itemize}
    \item Jedem Individuum ist es im Prinzip möglich, zu jedem Zeitpunkt seines Lebens in einer beliebigen Domäne Leistungsexzellenz zu erwerben
    \item Bedingung: Durchlaufen eines langjährigen, optimalen Trainingsprozesses
    \item Bedeutung der Genetik vernachlässigbar
    \item später partiell moderater
\end{itemize}

Die moderate Fassung ist teilweise vom Sport motiviert, da es einfach anthropometrische Vorrasusetzungen gibt, die weitgehen als genetisch determiniert gelten. Zusätzlich muss man den Altersgang der körperlichen Entwicklung berücksichtigen. Im z.B. Laufen wurde allerdings gezeigt, dass auch ``Späteinsteiger'' Nationale Spitzen erreichen können.

Deliberate Practice:
\begin{itemize}
    \item ``systematisches'' Training
    \item Prozess der Herausbildung von Leistungsexzellenz
    \item Ursprung: Domänenübergreifende Gemeinsamkeiten bei Karrieren von Spitzenleistern
    \item Merkmale:
    \begin{description}
        \item [Zeitlicher Umfang] sehr umfangreich, ``ten years rule'', domänenübergreifend ca 10.000 Trainingsstunden, im Sport Überbelastung?
        \item [Qualität des Trainings] entscheidende Komponente, optimale äußere Bedingungen und Zugriff auf Resourcen, geeignete Trainer, Alters- und Entwicklungsgemäß optimales Training
        \item [Krisen] Domäne wird ausschließliches Interessen- und Tätigkeitsfeld, Einschränkungen in anderen Lebensbereichen. Deliberate Practice bereitet nicht nur Freude!
    \end{description}
\end{itemize}

\paragraph{Phasen des Expertise-Erwerbs:}
In typischen Karriereverläufen kann man verschiedene Phasen erkennen in denen jeweils spezifische Entwicklungsaufgaben zu lösen sind. Sie unterscheiden sich bezüglich Ziele des Trainings, Widmung des Athleten, Institutionen des Trainings und Rolle der Trainer, Eltern und Freunden.
\begin{description}
    \item[Die frühen Jahre] Ziel: Erzeugen von Motivation und Bindung an die Domäne, hohes Maß an Freiwilligkeit. Trainer: Motivator, emotionale Kapazitäten, Familie spielt eine zentrale Rolle und Sport nimmt meist nur einen geringen Teil des Lebens ein.
    \item [Die mittleren Jahre] Vom Beginn der Aufnahme eines systematischen Trainings bis zur Professionalisierung des Trainings. Gut ausgebildete Trainer, Familie teils offizielle/inoffizielle Unterstützungsfunktion, sport determiniert sukzessive das restliche Leben, Dramatische Zunahme von Umfang und Intensität.
    \item [Die späten Jahre] Profi-Status, gesamtes Leben wird Exzellenzielen untergeordnet. Hohe Anforderungen an Trainer. Überdauernde Leistungsexzellenz stellt sich nach einigen Jahren in dieser Phase ein!
\end{description}

Probleme:
\begin{itemize}
    \item Langjähriger, grenzbelastender Prozess mit Krisen, zu leistenden Übergängen und deren Wechselwirkungen
    \item Oftmals verbunden mit erheblichen finanziellen Belastungen
    \item Externe Faktoren, welche nicht beeinflusst werden können
    \item Übergänge als Entwicklungsaufgaben bezüglich Trainer, Bildungseinrichtung, Familie und Freunde und Persönlichkeit
\end{itemize}

\subsection{Central Governour Theory}

Klassische Modelle der Ermüdung:
\begin{itemize}
    \item Sauerstoffdefizit im Muskel führt zu Laktatakkumulation führt zum Ende der Aktivität.
    \item Ermüdung ist eine Kathastrophe, die auf dem Versagen unserer Muskulatur beruht.
    \item Ermüdung tritt ein, wenn System an seine Grenzen stößt
    \item Katastrophen-Theorie postuliert Zusammenbruch des Gesamt-Systems
    \item aber wiedersprüchliche Befunde:
    \begin{itemize}
        \item Zusammenbruch bei Höhenbelastungen $\rightarrow$ Großteil der Systeme noch funktionsfähig
        \item Periphere Ermüdung: Übersäuerung führt zum Zusammenbruch $\rightarrow$ zwar Abfall der pH-Werte, aber Kraftproduktion nicht beeinträchtigt
        \item Zentralnervöse Ermüdung, d.h. ZNS kann nicht mehr ausreichende Impulse generieren $\rightarrow$ Rekrutierungsreserven unabhängig vom Ermüdungszustand vorhanden
    \end{itemize}
    \item offene Fragen: Warum gibt es Endspurts? Warum steigen Leistungen mit Zuschauern? Warum ist Motivation so wichtig?
\end{itemize}

Was ist Ermüdung?
Obige Argumente sprechen gegen „physiological event“, d.h. Ermüdung wird nicht peripher reguliert!
Man muss unterscheiden zwischen den physiologischen Symptomen und deren Wahrnehmung und Interpretation durch zentrale Einheiten.
Ermüdung ist eine Wahrnehmung, die sich komplex zusammensetzt aus physiologischem, biochemischem und anderem sensorischen Feedback aus der Peripherie.

\paragraph{Central Governor Theory:}
We regulate unconsciously our running speed to make sure that our hearts and brains have enough oxygen.
This means that there must be a “governor” who protects us from exercising so hard that we would damage ourselves.
Tim Noakes therefore called his theory “the  central governor model”.

At the start of the run, the unconscious brain chooses the pace by recruiting an appropriate number of fibres in the working muscles.
Its choice is determined by the expected length and difficulty of the run, and by a long list of other factors, including fitness, fatigue, temperature, motivation, presence of spectators or competitors...
At each step it further adapts the pace according to information coming from the conscious brain such as the distance already covered, or the true difficulty of the terrain.
It also uses information from the working muscles, such as the rate of glycogen consumption, and from other unconscious sources, e.g. amount of
oxygen in the blood, loss of fluid, rise in core temperature, etc...

In that way the brain aims to prevent catastrophic changes in the body.
It does so by reducing the amount of muscle fibers we are using and by making us feel tired.
According to this model, fatigue is not just a physical event but an opinion of the brain.
In other words: it is an emotion, which can be altered by other emotions and by physical factors.
This explains why we have some energy left for the final sprint, or to speed up when we hear friends and family shout our name.
A partnership between mind and matter!

Konsequenzen:
\begin{itemize}
    \item Neue Betrachtung von Ermüdung
    \item Neue Perspektive auf Training: Noch größere Bedeutung der Individualität, Psychologische Prozesse
    \item Neue Perspektive Forschung: Wie kann man CGT beweisen/prüfen? Wie lässt sich ein Gehirntraining absolvieren?
    \item Erweiterung von Regelwerken notwendig? Neues Dopingfeld? Athleten vor sich selbst schützen
\end{itemize}
