%!TEX root = ../report.tex

\section{Einführung}

\subsection{Trainingswissenschaft}
\begin{itemize}

\item Leistungsfähigkeit: zeitlich überdauernde Eigenschaften eines Individuums
  \begin{itemize}
  \item Konditionelle Komponenten: Ausdauer, Kraft, Schnelligkeit, Beweglichkeit
  \item Informationelle Komponenten: Technik, Taktik
  \item Determinanten: Einzelfaktoren, die die Ausprägung einer (Teil-) Komponente bestimmen
  \end{itemize}
\item Realisation: Aktion zur Bewältigung einer Aufgabe im Sport
\item Training: Maßnahmen zur Steuerung der Leistungsfähigkeit
  \begin{itemize}
    
  \item Training ist die planmäßige und systematische Realisation von Maßnahmen (Trainingsinhalte und Trainingsmethoden) zur nachhaltigen Erreichung von Zielen (Trainingsziele)
im und durch Sport.
  \item Beanspruchung -> Beanspruchungsfolgen -> Adaption \item Unterscheidung zwischen konditionell und informationell
  \item Belastung: objektiv von Außen auf den Organismus wirkende Faktoren
  \item Belastungsnormative: Größen zur Beschreibung des Belastungsreizes (Intensität, Dauer, Umfang,\ldots)
  \item Beanspruchung: individuelle Reaktion des Organismus auf eine Belastung (z.B. gemessen über Herzfrequenz)
  \item Trainingsziele: Ausgewählte Komponenten der Leistungsfähigkeit
  \item Trainingsmethoden: Kombination von Belastungsnormativen zum Ansprechen der Determinanten
  \item Trainingsinhalte: Übungsgut zur Implementierung der Methodik
  \item Der Anwendungskontext von Training, beeinflusst die Ziele und damit die Inhalte und Methoden
  \end{itemize}
\item Leistung: Relativierung von Aktion und -ergebnis (bzw.\ im Wettkampf) an einer Norm
  \begin{itemize}
  \item Leistungsdiagnostik
    \begin{enumerate}
      \item Beschreibung der Realisation
      \item Bewertung der Realisation vor dem Hintergrund einer Norm
      \item Rückschluss auf die Leistungsfähigkeit
      \item Ableitung von Trainingszielen
    \end{enumerate}
  \end{itemize}
\end{itemize}
